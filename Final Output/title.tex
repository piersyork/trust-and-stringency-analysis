\begin{titlepage}
\begin{center}
\vspace*{5cm}
\LARGE
\textbf{The Effect of Distrust on Government Stringency During the Covid-19 Pandemic}


\vspace{1.5cm}
\large
Candidate Number: KFHH0

\vspace{0.5cm}
Word Count: 8282

\vfill

Dissertation submitted in part-fulfilment of the Masters Course in Public Policy, UCL,  September 2021.

% \afterpage{\blankpage}
\end{center}
\end{titlepage}

\begin{center}
\textbf{Acknowledgements}

To my supervisor, Prof. Ben Lauderdale for his support and advice.

To my darling Annabelle for putting up with my ramblings about distrust.

To my parents for their constant support.
\pagenumbering{gobble} 
\end{center}

\pagebreak

\begin{center}
\vspace*{5cm}
\textbf{Abstract}
\end{center}
Since Covid-19 spread across the world, governments have sought to control it through imposing strict restrictions on their citizens, from work-place closures to stay at home orders and curfews. But the stringency of these responses has varied across countries, with some like Belarus having very few restrictions and others, like New Zealand, having incredibly strict responses. This paper explains these differences in response by treating the pandemic as a collective action problem and arguing that distrust in other people has played an important role in how countries attempt to solve this collective action problem, with higher levels of distrust creating a demand for intervention from the government. By using a random effects regression analysis, this paper finds that higher levels of distrust leads to higher levels of stringency during the pandemic. Further analysis finds that this effect has changed over the course of the pandemic, with a stronger effect as the pandemic has gone on.
\pagenumbering{gobble} 
\pagebreak